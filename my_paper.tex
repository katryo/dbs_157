
\documentclass[submit,techreq]{ipsj}
%\documentclass[submit,draft]{ipsj}


\usepackage[dvips]{graphicx}
\usepackage{latexsym}

\def\Underline{\setbox0\hbox\bgroup\let\\\endUnderline}
\def\endUnderline{\vphantom{y}\egroup\smash{\underline{\box0}}\\}
\def\|{\verb|}

\setcounter{巻数}{53}
\setcounter{号数}{10}
\setcounter{page}{1}

\受付{2011}{11}{4}
%\再受付{2011}{7}{16}   %省略可能
%\再再受付{2011}{7}{20} %省略可能
\採録{2011}{12}{1}

\begin{document}


\title{語の出現パターンと意味関係分析を用いた\\
Webからのタスク検索}

\etitle{Subtask search}

\affiliate{IPSJ}{情報処理学会\\
IPSJ, Chiyoda, Tokyo 101--0062, Japan}


\paffiliate{KU}{京都大学\\
Kyoto Uniersity}



\author{加藤 龍}{Ryo Kato}{IKU}[r.kato@dl.kuis.kyoto-u.ac.jp]
\author{大島 裕明}{Hiroaki Ohshima}{IPSJ}[ohshima@dl.kuis.kyoto-u.ac.jp]
\author{山本 岳洋}{Takehiro Yamamoto}{IPSJ,KU}[yamamot@dl.kuis.kyoto-u.ac.jp]
\author{田中 克己}{Katsumi Tanaka}{IPSJ,KU}[ktanaka@i.kyoto-u.ac.jp]
\author{加藤 誠}{Makoto P. Kato}{IPSJ,KU}[kato@dl.kuis.kyoto-u.ac.jp]

\begin{abstract}
本研究では, クエリとしてタスクが与えられた際に, そのタスクを達成するために必要なサブタスク群をWebから発見する手法を提案する. 提案手法では,動詞の含意関係や逆意関係にもとづいたルールによりクエリの変換を行う. 変換したクエリでWeb検索を行う. 得られたページから, タスク特有の言語パターンを用いて,サブタスク候補となるフレーズを発見する. 
\end{abstract}

\begin{jkeyword}
情報処理学会論文誌ジャーナル,\LaTeX,スタイルファイル,べからず集
\end{jkeyword}

\begin{eabstract}
This study is about subtask search.
\end{eabstract}

\begin{ekeyword}
IPSJ Journal, \LaTeX, style files, ``Dos and Dont's'' list
\end{ekeyword}

\maketitle

%1
\section{動機}

Web検索

Google
目的を

誕生以来, Webには多様な資料が集積され続けている. 情報をどう走査, 発見し, 取捨選択するか, 様々な方向からのアプローチがなされている. 
現在, いくつもの検索エンジンが実用化され, Webを舞台に情報発見や集約を行っている. だが, Webにおける情報検索は, いまだ多くの問題を抱えている. 
そのうちの一つとして, 目的を達成する手段を発見したいとき, できるだけ多くの方法を探そうとしても意外と見つけられないことがあげられる.

なにかを成し遂げたいが, どうすれば成功するかわからないとき, Web検索を使って実現方法を考える行為がよく行われている. たとえば「花粉症対策をする方法」をクエリに検索することで, 立体マスクをつける」や「アレロック錠剤を飲む」を発見することができる. だが, そうして発見できた「花粉対策をする方法」を採用すべきなのか, 簡単にはわからない. まだ発見できていない「花粉対策をする方法」のほうがよいかもしれないからだ.

こうした状況では, 多くのユーザーは「タスクを遂行するためにどんな方法があるか」をできるだけ多く発見するサブタスク検索を求める. 多様な答えが得られた段階で, 初めて, 安心して各方法を比較したり, 自分にとって最適な方法を考えることができるようになる.

サブタスク検索の例を説明する.たとえば「花粉症対策をする」というクエリを入力すると, 出力として


\begin{itemize}
\item \|立体マスクをつける|
\item \|アレロック錠剤を飲む|
\item \|医師の診断を受ける|
\item \|植物に近寄らない|
\end{itemize}


といったように, 複数の異なった選択肢を複数発見する.これがサブタスク検索の例である. このように, できるだけ多くの手法を探そうとする検索は現在の一般的な検索エンジンでは困難である. 花粉症対策の方法は非常に多様であり, 高くランクづけされたページであっても花粉症対策の方法のごく一部を含んでいるにすぎない. また高くランクづけされたページ同士は内容が重複していることが多く, 高ランクのページを見て回っても, 発見できる花粉症対策の方法の数は増えない.

本稿では可能な限り多様な方法をWebから発見する手法を提案し, 評価する.


%2
\section{関連研究}

\cite{yamatake}において, 広告を用いたクエリクラスタリングが試みられている. タスク遂行のための, 複雑な検索の手順を研究したものとして\cite{hassan}がある. また田麦らによるタスク遂行のためショッピングサイトなどサービスを提供するページを発見する研究\cite{tamugi}がある.

%3
\section{タスク関係の概要}
ある目的を達成するためにどのような順番でどのような行動を取るか考察する際, タスク, ゴール, アクションといった用語の定義が必要となる.


%3.1
\subsection{用語の定義}

本稿ではタスク, ゴールを以下のように定義する.

\begin{itemize}
\item \|ゴールとは, 目的の一部または全部を達成した状態である|
\item \|アクションとは, ある状態から別の状態に移行する行為である|
\item \|タスクとは, 遂行すると, 目的の一部あるいは全部を達成したことになるアクションである|
\item \|サブタスクとは, タスク同士の関係に着目したとき, 「遂行するともう一方のタスクの一部を遂行したことになる行動」である. このとき「もう一方のタスク」はスーパータスクである|
\end{itemize}


つまり, なんらかのゴールに至るアクションがタスクといえる.

サブタスクとスーパータスクは相対的な関係である. あるサブタスクは, 別のタスクをサブタスクだとして見ればスーパータスクにあたることもある.

例えば「花粉症対策をする」がスーパータスクだと仮定すると, サブタスクとして「アレロック錠を飲む」, 「マスクをつける」といったタスクがあげられる. ここで一段階視点を下げて「アレロック錠を飲む」をスーパータスクと捉えた場合, 「医師の診断を受ける」, 「薬局に行く」といった行動がサブタスクとなる. 逆の順番でいえば, 「薬局に行く」というサブタスクから見たとき「アレロック錠を飲む」はスーパータスクだが, 「アレロック錠を飲む」をサブタスクとした場合スーパータスクは「花粉症対策をする」となる.

ここで注意してほしい点がある. 「花粉症対策をする」のサブタスクとして「薬局に行く」もありうる. 二段階, 三段階下のタスクであってもそれはサブタスクなのである. 逆に, 「薬局に行く」のスーパータスクとして「花粉症対策をする」がある. つまり, 複数段階の連続的な関係と, サブ・スーパーの一段階の関係は両立できる.


%3.2
\subsection{タスク構造}
ある目的を達成するために行動を起こす際, どのような手順で状態が変異するのか整理をする. まず最終的な目的として「花粉対策をする」というタスクがあるとする. これは「花粉症対策をした」というゴールと等価である.

ユーザーは「花粉症対策をしていない」という状態から「花粉症対策をした」という状態に遷移するために必要なアクションを発見しようとする.

「花粉症対策をする」というタスクを遂行する方法はいくつも存在する. 「アレロック錠を飲む」, 「マスクをつける」, 「医師の診断を受ける」, 「薬局に行く」これらすべてが「花粉対策をする」のサブタスクである.

%4
\section{タスク検索の手法}

「花粉症対策をする」といったタスクをクエリとする.
「花粉症対策をした状態」をクエリとすることも考えられるが, 本稿ではそれは扱わない.

%5
\section{評価}
実験を行った.

%5.1
\subsection{実験の手法}
ベースラインとして

%5.2
\subsection{実験の結果}

%6
\section{考察}


%7
\section{結論}




\begin{thebibliography}{10}


\bibitem{yamatake}
The Widwom of Advertisers: Mining Subgoals via Query Clustering
\urlj{http://research.microsoft.com/en-us/people/tesakai/cikm2012yamamoto.pdf}
(2012.11.02).

\bibitem{hassan}
Task Tours: Helping Users Tackle Complex Search Tasks:
Ahmed Hassan, Ryen W. White
\urlj{http://research.microsoft.com/pubs/178868/HassanCIKM2012.pdf}

\bibitem{tamugi}
Tamugi search:
\urlj{http://somewhere.pdf}




\end{thebibliography}



\begin{biography}
\profile{m}{加藤 龍}{1988年生.2012年京都大学農学部卒. 2012年同大学院情報学研究科入学}
%
\end{biography}



\end{document}
